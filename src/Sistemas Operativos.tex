\documentclass[a4paper, twoside]{article}
\usepackage[utf8]{inputenc} % Especifica la codificación de caracteres de los documentos.
\usepackage[spanish]{babel} % Indica que el documento se escribirá en español.
\usepackage[top=3cm, bottom=2.5cm, inner=1.5cm, outer=2.5cm]{geometry} % Márgenes personalizados
\usepackage{subfiles} % Paquete para incluir el preambulo en los sub archivos.
\usepackage{afterpage} % Permite añadir páginas despues de una página dada.
\usepackage{hyperref} % Permite incluir enlaces en los archivos.
\usepackage{lastpage} % Paquete para poder contabilizar el total de páginas del documento.
\usepackage{fancyhdr} % Permite personalizar los header y footer del documento.
\usepackage{graphicx} % Permite incluir gráficos
\usepackage[hang, bf]{caption} % Personaliza los subtítulos de las figuras y tablas
\usepackage{float} % Permite posicionar mejor las figuras y tablas

% Defino la ruta de los paquetes personalizados para el apunte
\newcommand{\rutapaquetes}{./paquetes-apunte}

\usepackage[mostrarlicencia]{\rutapaquetes/caratula} % Caratula personalizada (cargada desde caratula.sty)
\usepackage[mostrarrevisores]{\rutapaquetes/colaboradores} % Seccion de colaboradores (cargada y creada con colaboradores.sty)
\usepackage{\rutapaquetes/historial} % Seccion de historial de cambios (cargada y creada con historial.sty)

% Define los estilos de los enlaces interpretados por el paquete hyperref
\hypersetup{
	colorlinks=true,   % false: boxed links; true: colored links
	linkcolor=black,   % color of internal links (change box color with linkbordercolor)
	citecolor=green,   % color of links to bibliography
	filecolor=magenta, % color of file links
	urlcolor=blue,     % color of external links
}

\newcommand{\imgdir}{../resources/images} % Ruta de las imágenes

% Define los directorios de las imágenes y gráficos
\graphicspath{ {\imgdir/} {\rutapaquetes/} }

\newcommand{\nombremateria}{Sistemas Operativos (75.08 - 95.03)} % Defino el comando "\nombremateria" para no harcodear el nombre en varios lugares.

% Define el pagestyle personalizado
\pagestyle{fancy}
\fancyhf{}
\renewcommand{\sectionmark}[1]{\markboth{}{\thesection\ \ #1}}
% Define header para pagina par
\fancyhead[ER]{\rightmark}
% Define header para pagina impar
\fancyhead[OL]{\rightmark}
% Define footer para pagina par
\fancyfoot[EL]{\nombremateria} % Nombre del apunte a la izquierda
\fancyfoot[ER]{Página \thepage\ de \pageref{LastPage}} % Numero de pagina a la derecha
% Define footer para pagina impar
\fancyfoot[OL]{Página \thepage\ de \pageref{LastPage}} % Numero de pagina a la izquierda
\fancyfoot[OR]{\nombremateria} % Nombre del apunte a la derecha

\renewcommand{\footrulewidth}{0.4pt} % Agrego linea que separa el footer

% Configura la caratula
\materia{\nombremateria}
\tipoapunte{Resumen teórico}
%\tema{Tema de la Materia}
%\subtema{Subtema}

\begin{document}
% Página en blanco agregada después de la carátula
%\afterpage{
%	\null
%	\thispagestyle{empty}%
%	\addtocounter{page}{-1}%
%	\newpage}
\maketitle % Genera la carátula

\tableofcontents % Genera el índice

\subfile{\rutapaquetes/acerca-del-proyecto.tex} % Inlcuye informacion acerca del proyecto FIUBA Apuntes

% Insertar aquí el contenido del apunte. A continuacion hay secciones a modo de ejemplo.

\section{Introducción}
\subsection{¿Qué es un sistema operativo?}
\begin{itemize}
	\item Un programa que hace de intermediario entre el usuario de la computadora y su hardware (Oculta los detalles finos de la arquitectura).
	\item Un programa que administra los recursos de un sistema de computación: permite administrar el tiempo de procesador y el espacio (memoria, disco, etc).
\end{itemize}

\subsection{Arquitecturas}
\subsubsection{Mainframe}
Computadora central. Gran capacidad de I/O, server para e-commerce a gran escala.\\

\textbf{Seguridad y disponibilidad:} 
\begin{itemize}
	\item Transaction processing: is information processing that is divided into individual, indivisible operations, called transactions. Each transaction must succeed or fail as a complete unit; it can never be only partially complete.
	\item Batch processing: is the execution of a series of programs ("jobs") on a computer without manual intervention.
\end{itemize}

\subsubsection{Servidores}
Destinados a ofrecer servicios a través de una red.

\subsubsection{Supercomputadoras}
Computacion de alto rendimiento. Se usan para hacer simulaciones.\\

\textbf{Limites:}
\begin{itemize}
	\item Concurrencia: los procesos no son 100\% independientes
	\item Costo
	\item Programación del software
\end{itemize}

\subsubsection{Server operating system}
Interfaz solo línea de comando o EFI (estándar de firmware).

\subsubsection{Computadora personal}
No requiere conocimientos especiales.

\subsubsection{Tablets, PDA}

\subsubsection{Consolas}

\subsubsection{Sistemas operativos embebidos}
Dispositivos que no aceptan instalación de nuevo software por el usuario.

No deberían tener bugs.

Se usan en tvs, autos, etc.

\subsubsection{Cluster}
Un grupo de computadoras interconectadas por una red local de alta velocidad.

Se comportan como si fuese una única computadora.

Si es de alta disponibilidad tiene nodos redundantes en caso de falla. Retoma en otro equipo en el estado en el que estaba. 

Balance de carga, con dispositivo físico o de software.

\subsubsection{Grid}
Cluster virtual con recursos distribuidos.

\textbf{Ejemplo:} BOINC, SETI.

\textbf{Problemas:} concurrencia (que se choquen tareas), que queden tareas sin cubrir.

\subsubsection{Cloud computing}
Se provee por internet. Dinamicamente escalable.

Atrás de la nube puede haber cluster, grid, etc (al cliente no le importa).

\textbf{Servicios posibles de cloud computing:}
\begin{itemize}
	\item Cloud Storage: Dropbox.
	\item Infraestructura (infraestructura as a service (IaaS)):Tipicamente plataformas virtualizadas. Ejemplo: Amazon EC2
	\item Plataforma (PaaS): Provee la plataforma y un ambiente de desarrollo y soporte. Ejemplo: Google Code.
	\item Software (SaaS): Software on demand provisto por terceros. Ejemplo: Amazon Services, Paypal. 
\end{itemize}

\subsubsection{Tiempo real}
Distinto de online o de rápido.

Tiempo de respuesta máximo y predecible. 

\subsubsection{Multiprocesador}
Más de un procesador en el mismo chip o board.

Soportado en todos los sistemas operativos de escritorio.

La paralelizacion esta limitada por la ley de Amdahl: The speedup of a program using multiple processors in parallel computing is limited by the time needed for the sequential fraction of the program.

\newpage
\section{Mecanismos básicos}

% Bibliografía utilizada en el apunte
\newpage
\newcommand{\bibliographyname}{Bibliografía} % Defino el nombre de la sección de la bibliografía
\addcontentsline{toc}{section}{\bibliographyname} % Agrego la bibliografía en el índice
\renewcommand\refname{\bibliographyname} % Renombro a la bibliografía (por default es 'Referencias')
\begin{thebibliography}{X}
	\bibitem{tanenbaum} \textsc{Andrew S. Tanenbaum}, \textit{Sistemas Operativos Modernos}, tercera edición, PEARSON EDUCACIÓN, México, 2009.
\end{thebibliography}

% Incluir los nombres de las personas que han colaborado en la creación del apunte
%\colaborador{Colaborador 1}
%\colaborador{Colaborador 2}
%\revisor{Dr. Profesor}{10/01/2015}
%\makeseccioncolaboradores % Crea la seccion de colaboradres

% Incluir el historial de cambios
\revision{21/02/2012}{Versión inicial con las secciones de Introduccion, Mecanismos básicos.}
%\revision{10/01/2015}{Se agregó una carátula personalizada, una sección con información del proyecto FIUBA Apuntes, la sección de colaboradores del apunte, sección de historial de cambios, ejemplo de bibliografía, ajuste de márgenes, encabezado y pie de página.}
\makehistorial

\end{document}